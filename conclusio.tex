\chapter{Conclusio}

%------------------------------


\section{ Zusammenfassung der Resultate}
\label{sec:4 conclusio}

Ziel dieses wissenschafltichen Versuches war es, einen  Steiner Baums\cite{jungnickel1} zu approximieren.
Hierfür wurde ein minimal spannenden Baum \cite{krumke1} konstruiert und die Resultate beider Ansätze wurden abschließend verglichen.

\section{ Interpretation der Ergebnisse}
\label{sec:4 conclusio}

\par  Die Untersuchungen ergaben, dass es mit dem Ansatz des MST zu goßen Abweichungen kommen kann.
Die erhaltenen Versorgungslängen verhielten sich im Durchschnitt um 61,029\% länger, als 
die Versorgungslängen welche mittels Steiner Baum errechnet wurden. Dies ist in Abbildung 3.2 ersichtlich.

\vspace{0.5cm}

Nachfolgend wurde eine statistische Auswertung, unter Ernennung eines Konfidenzintervalls durchgeführt.
Von den ursprünglichen Wählamtsbereichen, welche für die Untersuchung herangezogen wurden, lagen $46\%$ im berechneten Intervall.
Mit einem Ergebnis, dass sich mehr als die Hälfte der untersuchten Gebiete nicht in das Modell aussagekräftig einbinden lassen und somit nicht
als Vergleichswert herangezogen werden konnten.\\







