\documentclass[oneside,bachelor,etd]{BYUPhys}
%\usepackage[latin1]{inputenc}
\usepackage{ifpdf}
\ifpdf
  \usepackage[pdftex]{graphicx}
\else
  \usepackage[dvips]{graphicx}
\fi
\usepackage[german]{babel}
\usepackage[utf8]{inputenc}
\usepackage[T1]{fontenc}
\usepackage{anysize}
\marginsize{2.5cm}{1.5cm}{1.5cm}{1.5cm}






\usepackage{fancyhdr}
  \fancyhead{}
  \fancyhead[LO]{\slshape \rightmark}
  \fancyhead[RO,LE]{\textbf{\thepage}}
  \fancyhead[RE]{\slshape \leftmark}
  \fancyfoot{}
  \pagestyle{fancy}
  \renewcommand{\chaptermark}[1]{\markboth{\chaptername \ \thechapter \ \ #1}{}}
  \renewcommand{\sectionmark}[1]{\markright{\thesection \ \ #1}}

\usepackage[margin=0.3in,labelfont=bf,labelsep=none]{caption}
 \DeclareCaptionFormat{suggested}{\singlespace#1#2 #3\par\doublespace}
 \captionsetup{format=suggested}

\usepackage{cite}


\usepackage{algorithm}
\usepackage{algorithmic}

\usepackage{lmodern,dsfont}


\usepackage{makeidx}
\makeindex

\usepackage{url}
\urlstyle{rm}

\usepackage[bookmarksnumbered,pdfpagelabels=true,plainpages=false,colorlinks=true,
            linkcolor=black,citecolor=black,urlcolor=blue]{hyperref}

\usepackage{float}
\usepackage{color}
\usepackage{xcolor}
\usepackage{listings}
\usepackage{caption}

\definecolor{MyGray}{rgb}{0.91,0.92,0.93}
\usepackage{calc}
\newsavebox{\fcolbox} \newlength{\fcolwidth}
\newenvironment{graybox}[1][\linewidth]
	{\setlength{\fcolwidth}{#1-2\fboxsep-2\fboxrule}%
	 \begin{lrbox}{\fcolbox}\begin{minipage}{\fcolwidth}}
	{\end{minipage}\end{lrbox} \noindent \colorbox{MyGray}{\usebox{\fcolbox}}}



\usepackage{color}
\usepackage{xcolor}
\definecolor{lightgrey}{gray}{0.95}

\usepackage{listings}
\lstset{
	language=C++,
	basicstyle=\footnotesize,
	backgroundcolor=\color{lightgrey},
	rulesepcolor=\color{gray},
	tabsize=2,
	breaklines=true,
	morekeywords={all},
	literate=%
{Ö}{{\"O}}1 
{Ä}{{\"A}}1 
{Ü}{{\"U}}1 
{ß}{{\ss}}2 
{ü}{{\"u}}1 
{ä}{{\"a}}1 
{ö}{{\"o}}1
}
\DeclareCaptionFont{white}{\color{white}}

\renewcommand*{\lstlistingname}{Algorithmus}

%\DeclareCaptionFont{white}{\color{white}}
%\DeclareCaptionFormat{listing}{\colorbox{gray}{#1#2#3}}
%\captionsetup[lstlisting]{format=listing,labelfont=white,textfont=white}


\definecolor{hellgrau}{gray}{.9}
\usepackage{amssymb}
\DeclareCaptionFont{white}{\color{white}}
\DeclareCaptionFormat{listing}{\colorbox{gray}{\parbox{\linewidth+5.5\fboxsep}{#1#2#3}}}
\captionsetup[lstlisting]{format=listing,labelfont=white,textfont=white}
\usepackage{amsmath}

% ------------------------- Fill in these fields for the preliminary pages ----------------------------
  \Year{2011}
  \Month{January}
  \Author{Dietmar Pichler}
  \TitleTop{Beschränkung von Clustern }
 \TitleBottom{ Eine Berechnungsgrundlage des Routingmodells}

\Abstract{

In dieser Arbeit wird ein Modell beschrieben, welches ein bestehendes und bereits implementiertes Programm (RTR\_R2008a) zu einer verbesserte
Performance
führen soll.
Das Programm wurde von FH-Prof. Dipl.-Ing. Dr. Peter Bachhiesl geschrieben und berechnet eine Netzwerkinfrastruktur von Grund auf neu. 
Die Netzwerkinfrastruktur bildet das Telekommunikationsnetz der Telekom Austria ab und ermittelt die Längen der Kabel bezüglich der Trassenführung.
Diese Längen, werden Versorgungslängen genannt.\\
Es wird angestrebt, dass bestehende Programm hinsichtlich der maximalen Versorgungslängen, bereits in einem frühen Stadium der Modellierung
einzugrenzen. Somit kann in weiterer Folge sichergestellt werden, dass keine unzulässigen Versorgungslängen im Modell weiter berechnet werden.
Werden diese Längen nicht eingeschränkt, kann es passieren, dass im letzten Schritt der Modellierung zu einer Fehlkalkulation kommt.\\
Daraus resultierend, muss der gesamte Vorgang der Modellierung nochmals durchlaufen werden. Es kann nicht ausgeschlossen werden, dass wiederum ein Fehler
auftritt.
Dieses Ergebnis soll erreicht werden, durch eine Implementierung eines Testbeds.  
Darum wird das Ziel verfolgt, dass jeder Modellierungsvorgang exakte Resultate liefert.
 


\vspace{0.5cm}

\textbf{Schlüsselwörter:} Versorgungslängen, Kanten, Knoten, Steiner Baum, minimal spannender Baum, Algorithmen

\vspace*{1cm}
 
This thesis describes a model which has already been implemented by FH-Prof. Dipl.-Ing. Dr. Peter Bachhiesl. 
The program simulates a new data network which is based on an existing network infrastructure from the Telekom Austria. 
However a few weaknesses are included.
The main aim of this employment is to rise the performance by purging the weaknesses. 
In fact, it is to find the maximum length from the distribution center to the customer.
The goal is to isolate higher-than-average lengths in an early part of the modeling process. 
This to make sure that every calculation is able to provide correct results.
As a result of not limiting these lengths it could occur that the whole procedure has to be continued.
To reach this outcome, a testbed has to be implemented.
Therefore, every single modeling process will deliver exact findings.
\vspace{0.5cm}

\textbf{keywords:} distribution lengths, edges, nodes, Steiner tree, minimal spanning tree, algorithms

}

% ------------- These remaining fields are only necessary for masters and PhD ----------------------

\begin{document}

 % Start page counting in roman numerals
 \frontmatter

\begin{titlepage}
\selectlanguage{german}
\hspace{8cm}
\begin{minipage}[b]{1cm}
%\begin{figure}[h]
\begin{center}
% bei der Bearbeitung mit LATEX
%\includegraphics[width=3cm]{fhlogo0.png}
% bei der Bearbeitung mit pdfLATEX
\includegraphics [scale=0.3]{pics/fhlogo0.png}
%\epsfig{file=fh.eps, height=7cm}
%\caption{Zerfall}\label{fig:ak1}
\end{center}
%\end{figure}
\end{minipage}
%\hspace{-9.85cm}
\hspace{-10cm}
\textbf{Fachhochschulstudiengang\\
Telematik / Netzwerktechnik}
\vspace{2cm}
\begin{center}
{\huge \bfseries B A C H E L O R A R B E I T}\\
\vspace{3.2cm}
\Large{\textbf{Beschr\"ankung von Clustern}}\\
\Large{Eine Berechnungsgrundlage des Routingmodells}\\
\vspace{2.4cm} \normalsize zur Erlangung des akademischen Grades\\
Bachelor of Science
\end{center}
\vspace{1.5cm}
%
%\begin{figure}[h]
%\begin{center}
%\includegraphics[height=7cm]{hdynam3.eps}
%\epsfig{file=hdynam3.EPS,height=5cm}
%\caption{?bertragungsrate und
%\end{center}
%\end{figure}
%
\begin{tabular}{l l}
Autor: & Dietmar Pichler\\
Matrikelnummer: & 0810286040\\
& \\
& \\
Erstbetreuer: & FH-Prof. Dipl.-Ing. Dr. Peter Bachhiesl\\
& \\
Zweitbetreuer: & \\
& \\
& \\
& \\
Tag der Abgabe: & \\
%
% Version 0.5 am 8. Oktober 2004
%
\end{tabular}
%\begin{table}
\end{titlepage}

\input{eidesstatth}

 % This command makes the formal preliminary pages.
 % You can comment it out during the drafting process if you want to save paper.
\makepreliminarypages

 \singlespace

 % Make the table of contents.
 \tableofcontents
 \clearemptydoublepage

 % Make the list of figures
 \listoffigures
 \clearemptydoublepage

 \doublespace

 % Start regular page counting at page 1
 \mainmatter

% --------------------------------------------------------------------------



\chapter{Einleitung}
%=================================================

\section{Allgemeines}
\label{sec:1einleitung}


\vspace{0.5cm}


\par Aufgrund des täglich steigenden Datenaufkommens und der damit verbundenen benötigten Infrastruktur, besteht ein Wettbewerb zwischen 
den Datennetz-Anbietern. Existierende Datennetze werden von Errichtern vielfach auch an konkurrierende Anbieter vermietet, da es nicht lukrativ wäre, wenn
jeder Anbieter sein eigenes Netzwerk errichten würde. Hierfür werden Mietgebühren verlangt, welche von der Austrian Regulation Authority for Broadcasting 
and Telecommunications (RTR), festgelegt werden. 
Da die tatsächlichen Kosten der bereits bestehenden Infrastruktur, die für diese Verkabelung von der Telekom Austria angefallen sind, nicht mehr nachvollziehbar 
sind, werden Modelle benötigt, welche eine völlig neue Verkabelung oder verschiedene Szenarien eines Ausbaues simulieren, um einen fairen Preis festlegen
zu können.
Bislang existierende Modelle besitzen einen hohen Abstraktionsgrad, das bedeutet, dass die Ergebnisse nur schwer in die realen Gegebenheiten
übertragen werden können. Deshalb sind sie weder für Netzwerk-Errichter noch für Regulatoren von großem Nutzen, da sie nicht für eine 
Grobplanung oder eine Modellierung herangezogen werden können. Die Ergebnisse solcher Programme sollten für die Festlegung der Mietkosten herangezogen
werden können oder einer Investitionsabschätzung dienlich sein.
\par Darum wurde ein Programm, namens RTR\_R2008a, entwickelt und implementiert. Dieses liefert Ergebnisse, die den realen Umständen nahekommen.
Jedoch beinhaltet das Programm noch Schwachstellen, wie zum Beispiel eine nicht vorhandene Begrenzung der maximal zulässigen Kabellängen, die es zu 
beseitigen gilt\cite{pinkafeld1}. 

%=================================================
\section{Umfeld}
\label{sec:1einleitung}


\vspace{0.5cm}

\par Das Programm RTR\_R2008a unterteilt Wählamtsbereiche in Sub-Bereiche. Ein Wählamts-bereich der Telekom Austria ist ein geographisches Gebiet, 
welches von einem Verteilzentrum, dem Wählamt, Endteilnehmer mit Datenleitungen versorgt. 
\\Die Größenbestimmung dieser Sub-Bereiche geschieht in
einem parametrierbaren Cluster-Modell des Programms. 
Diese Parameter sind der maximale/absolute Informations(gehalt) bzw. -bedarf und die maximale/absolute Dilatation.
Der Informationsbedarf beschreibt die Anbindung an das Netzwerk. Diese kann mit Kupfer-oder Lichtwellenleitern
erfolgen. Wird die Anbindung mit Kupferkabeln vorgenommen, entspricht der Informationsbedarf der Anzahl der benötigten Kupferleiter. 
Werden Glasfasern verwendet, so gibt der Informationsbedarf die Anzahl der benötigten Wellenlängen eines Lichtwellenleiters wieder. Bestimmt wird 
der Informationsgehalt auf Basis der Mikrozellendaten. 

\vspace{0.5cm}

Die Dilatation eines Wählamtsbereiches ist die maximale, euklidische Distanz, über alle möglichen
Anschlussobjekt-Paarungen.
Nicht berücksichtigt wird die maximal mögliche Versorgungslänge von der Quelle (Distribution Center) zur Senke (Anschlussobjekt). 
Es besteht ein Zusammenhang zwischen Parametereinstellungen im Cluster Modell und der daraus resultierenden, maximalen Versorgungslänge in den 
Sub-Bereichen.
Jedoch können über entsprechende Modellparameter keine Einschränkungen bezüglich der maximal möglichen Versorgungslängen im Cluster Modell getätigt
werden\cite{tech_rep_1}.

Es wird ein Abbild der Trassenführung mittels eines Objektes, bestehend aus Knoten und Kanten erzeugt.  Dieses Objekt ist ein Graph, 
welcher die Wählamtsbereiche der Telekom\\ Austria simuliert und die entsprechenden Anschlussobjekte
aufzeigt\cite{pinkafeld1}.
\\

Um das bestehende System zu optimieren, sollten die Beschränkung der maximalen Anschlusslängen bereits im Clustermodell getroffen und nicht erst 
im rechenintensiven Routingmodell durchgeführt werden. 

%=================================================
\section{Prozessablauf}
\label{sec:1einleitung}


\vspace{0.5cm}


Der Ablauf ist in mehrere Schritte und Teil-Modelle gegliedert. Die Informationen für folgendes Kapitel wurden aus \cite{pinkafeld1 } entnommen,
sofern nicht anders angegeben. 

\subsubsection{Geodaten}
Vorab muss eine Aufbereitung der Geodaten\cite{prossegger1} erfolgen. Diese werden in einem Graphen dargestellt und als $.ist$ Dateien abgespeichert. 
Sie dienen dem Clustermodell als Input.
\subsubsection{Clustermodell}

Im nächsten Schritt wird der Netzwerkgraph (Wählamtsbereich) in Teilgraphen aufgespalten um unterschiedliche Hierarchie-Ebenen zu erhalten. 
Dieser Prozess wird als \textit{clustern} bezeichnet und im gleichnamigen Modell durchgeführt (Kapitel 2.1.2).  
\vspace{0.3cm}

Als Ergebnis erhalten wir unterschiedliche Hierarchieebenen:

\begin{description}
\item[Citynetzebene:] Entspricht dem Wählamtsbereich (gesamter Netzwerkgraph)
\item[Accessnetzebene:] Teilgebiet des gesamten Wählamtsbereich 
\item[Accesszellebene:] Gibt Gebiet der anzubindenden Endteilnehmer zu den Netzwerkverteilern wieder
\end{description}

\vspace{0.5cm}

Die erhaltenen Ebenen sind wie folgt miteinander verbunden:


\begin{description}
\item[Citynetzebene und Accessnetzebene:] Das Wählamt wird mit Verteilern in der Accessnetz-ebene (Distribution Center) verbunden
\item[Access(netz)-und Zellebene:] Ist die Verbindung von Distribution Center zu den Endeinrichtungen, auch als \textit{last mile} bezeichnet
\end{description}

\subsubsection{Routingmodell}


Der Datenoutput des Clustermodells dient als Input für das Routing Modell (Kapitel 2.1.4). In diesem werden die tatsächlichen Wege für die Trassenführung
aufgrund der Faktoren Flächennutzung und Bebauungsgrad nun abstrahiert und dargestellt.
\\
Der Output des Routingmodells bildet Trassenwege inklusive Anschlussobjekte ab, welche unter Rücksichtname den optimalen Kosten genügen.
Des weiteren wird eine Auflistung der Kosten und Mengen zum Beispiel der benötigten Leitungen generiert, die als Grundlage einer Investitionsabschätzung dienen
\cite{pinkafeld1}.
\\
Derzeit kann es passieren, dass sich erst nach einem gesamten Modellierungsdurchlauf eine überdurchschnittlich lange Versorgungslänge ergibt. 
 


%=================================================
\section{Aufgabenstellung}
\label{sec:1einleitung}


\vspace{0.5cm}

In dieser Arbeit wird  eine Abschätzung getroffen,  bei welcher die maximale Versorgungslänge noch vor dem Routing Modell passiert. 
Bisher werden im Clustermodell die maximalen Anschlusslängen nicht direkt, sondern über den Informationsbedarf und die maximale Dilatation eingeschränkt.
Gezeigt wurde dies im Projekt \textit{Versorgungslängensensitivität} Semester IV.
Es wird eine Implementierung benötigt, welche die maximale Anschlusslänge unmittelbar nach dem Clustermodell ermittelt und dem Routingmodell einen
geeigneten Input liefert. 
\\
\textit{Ziel} ist es, vermeintliche Fehler bezüglich der Anschlusslängen bereits vor dem rechenintensivem Routingmodell zu erkennen und diese zu vermeiden.
 Das ist ein weiterer Schritt, um das bestehende Programm zu optimieren.
Somit ist sichergestellt, dass jede Modellierung korrekte Ergebnisse liefert.



% --------------------------------------------------------------------------

\input{Methoden}

% --------------------------------------------------------------------------

\input{Ergebnisse}

% --------------------------------------------------------------------------

\input{Conclusio}

% --------------------------------------------------------------------------


% Start labeling chapters with letters and calling them appendices
\appendix

 % -----------------------Make the bibliography-------------------------------
\cleardoublepage
 \phantomsection \addcontentsline{toc}{chapter}{Literaturverzeichnis}
 \begin{thebibliography}{00}
 
\bibitem{pinkafeld1}
 Bachhiesl,P.,et al;{\em  Bottom up Kostenmodellierung von Hybridnetzen }
 
\bibitem{tech_rep_1}
 Bachhiesl,P.,et al; {\em Model-and parameter description for the bottom-up network simulaion tool RTR\_R2008a-Technical Report}, (Peter Bachhiesl)

\bibitem{bachhiesl2}
 Bachhiesl,P.; {\em Skriptum zur Lehrveranstaltung Mathematik III}, (Peter Bachhiesl)
 
 \bibitem{bronstein}
Bronstein,et al; {\em Taschenbuch der Mathematik},(Bronstein,) 7.Auflage, Frankfurt am Main, 2008


 \bibitem{jungnickel1}
Jungnickel, D.; {\em Graphs networks and algorithms}, 3. Auflage, Berlin Heidelberg, 2008


\bibitem{krumke1}
 Krumke,S.; Noltemeier, H.; {\em Graphentheoretische Konzepte und Algorithmen}, 2. Auflage, Wiesbaden, 2009
  
\bibitem{papageorgiou}
Papageorgiou,M.; {\em Optimierung}, 2. Auflage, München, 1996

\bibitem{prossegger1}
Paulus,G., et al; {\em Entwicklung von kostenoptimierten räumlichen Szenarien für den strategischen Ausbau der Glasfasernetz-Infrastruktur am Beispiel eines Multi-Utility Unternehmens}, \ 2. Forschungsforum der österreichischen FHs,\ FFH 2008, \ 26-27 Mär 2008, \ Wels, Austria.
 
\bibitem{turau1}
 Turau,V.; {\em Algorithmische Graphentheorie}, (Volker Turau), 3. Auflage, München, 2009


\end{thebibliography}


%----------------------------------appendix-----------------------------------

%
%\chapter{Stichwortverzeichnis}
%\label{sec:appendix}
%
%\subsection{Begriffserklärungen}
%
%
%
%
%\begin{flushleft}
%distribution center - Versorgungszentrum in der jeweiligen Hierarchie-Ebene 
%\end{flushleft}
%
%\begin{flushleft}
%euklidische Distanz - der  Abstand zweier Punkte einer Ebene oder eines Raumes
%\end{flushleft}
%
%\begin{flushleft}
%heuristisch: [Mathematik] unter geringem Aufwand kein optimales, aber ein meist brauchbares Ergebnis liefernd
%\end{flushleft}
%
%\begin{flushleft}
%Kanten - tatsächliche Verlegetrassen oder paarweise Verbindungen welche Knoten miteinander verbinden
%\end{flushleft}
%
%\begin{flushleft}
%Knoten - Anschlussobjekte oder Verbindungsknoten
%\end{flushleft}
%
%\begin{flushleft}
%knotendisjunkt: Wege, Pfade, Zyklen und Kreise in Graphen
%\end{flushleft}
%
%\begin{flushleft}
%spatiale Knoten - Knoten die sich aufgrund der Graphenkonstruktion ergeben, sind Verbindungsknoten ohne Informationsgehalt
%\end{flushleft}
%
%\begin{flushleft}
%Versorgungslänge in Accesszellebene- beschreibt die Distanz eines Endteilnehmers, bis zu 
%den ihn sich am nächsten befindenden Versorgungszentrum (distribution center)
%\end{flushleft}



%	\$ git status
%	# On branch master
%	nothing to commit (working directory clean)
%	\$ git branch experimental
%	$ git checkout experimental 
%	Switched to branch 'experimental'
%	$ edit files ...
%	git commit -a
%	[experimental f4e4220] edit main
% 	1 files changed, 7 insertions(+), 0 deletions(-)
%	$ git checkout master
%	Switched to branch 'master'
%	$ git merge experimental
%	Updating ed66c24..f4e4220
%	Fast-forward
%	main.c |    7 +++++++
% 	1 files changed, 7 insertions(+), 0 deletions(-)
%	$ git branch -d experimental 
%	Deleted branch experimental (was f4e4220).
%\end{lstlisting}


 % Make the index
 \cleardoublepage
 \singlespace
 % NOTE: '\phantomsection' helps get the pdf bookmark to work right. You need
 % to put it before every manual addition to the table of contents
 \phantomsection
 \addcontentsline{toc}{chapter}{Index}
 \printindex

\end{document}
